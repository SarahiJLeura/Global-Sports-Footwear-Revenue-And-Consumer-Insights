\documentclass[12pt]{article}

% Encoding
\usepackage[utf8]{inputenc}

% Page layout
\usepackage[margin=1.25in]{geometry}
\usepackage{setspace}

% Graphics and figures
\usepackage{graphicx}
\usepackage{subcaption}
\graphicspath{{./figures/}}

% Math and tables
\usepackage{amsmath}
\usepackage{booktabs}

% Links
\usepackage{hyperref}

% Bibliography
\usepackage[numbers]{natbib}

% Line spacing
\setstretch{1.15}

%%%%%% Title %%%%%%
\title{\textbf{Predicting Revenue and Analyzing Consumer Behavior in Global Sports Footwear Sales (2018--2026)}}

%%%%%% Autor %%%%%%
\author{
Jimenez Leura Patricia Sarahi \\
Student ID: 355963 \\
Intelligent Systems Engineering \\
Engineering School, Computer Science Department \\
Autonomous University of San Luis Potosi \\
Data Science -- Group 280801 \\
Professor: PhD.\ Jose Ignacio Nuñez Varela
}

\date{February 2026}

\begin{document}

\maketitle
\thispagestyle{empty}   % Removes page number from cover
\newpage

% Section 1: Overview of project
\section{Introduction}
Your introduction starts here \cite{footwear2026}.

\subsection{Problem description}
Explain what you are trying to analyze; a general description, but it
should provide a good sense of what the problem is. Also, it should contain the description of the task to be solved (i.e., classification, regression, clustering, etc.).

\subsection{Domain explanation}
Depending on the project, you may need to explain the domain or some background of the project so that it is understandable.

\subsection{KDD methodology}
Brief description of the Knowledge Discovery in Databases steps.

% Section 2: Dataset
\section{Dataset}


\subsection{Data selection}
Description of how the data set to be used was obtained and selected.

\subsection{Dataset description}
In-depth description of the dataset to be used.

\subsection{Attributes description}
List of all attributes, their description, and values.

\subsection{Statistical analysis}
Analysis of the dataset in general, and for each attribute.

\subsection{Final comments}
Conclusions about the analysis made on the dataset.

% References

\bibliographystyle{plainnat}
\bibliography{references}

\end{document}
